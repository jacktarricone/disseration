
%==============================================================================2
%==============================================================================
%==============================================================================
%==============================================================================
\hypertarget{ch1}{%
\chapter{Introduction}\label{ch1}}


It is widely known that snow is a critical water resource for the western United States. However, our ability to accurately measure and monitor changes in snowpack, specifically its water equivalent, is challenged in mountain regions. Station-based measurements of snow water equivalent (SWE) are sparse, and spaceborne remote sensing does not yet have the capacity to directly measure SWE at the spatial resolution needed for water management applications. The overarching goal of this dissertation research is to progress our ability to track changes in SWE at a range of spatial and temporal scales. We will use two research themes to accomplish this goal: testing the feasibility of L-band interferometric synthetic aperture radar (InSAR) to measure SWE changes and conduct a novel geospatial analysis of climatic-scale SWE trends in the California Sierra Nevada. While these two methods are distinctive, they both work towards forwarding our ability to understand and monitor the dynamics of mountain snowpack. \par

In the western US, the majority of precipitation in mountainous regions falls in the winter as snow \citep{serrezeCharacteristicsLargeSnowfall2001}, where it is stored in the mountain snowpack “water towers” until it begins to melt in the spring (Immerzeel et al., 2020; Viviroli et al., 2007). Snowmelt runoff provides as much as 70 \% of the total water for areas near mountains (Li et al., 2017) while filling reservoirs, irrigating agricultural fields, and providing habitat for aquatic species (Yarnell et al., 2010).
The need to measure the distribution of snow water equivalent (SWE) from space has been well documented for decades within the hydrologic sciences community (Lettenmaier et al., 2015; Peters-Lidard et al., 2018). The most recent NASA Decadal Survey listed snow depth and SWE as target observables for future NASA missions and essential for both water resource forecasting and global land surface energy balance modeling (Space Studies Board et al., 2019). The need for these measurements becomes increasingly important as the areal and temporal extent of snowpacks are expected to decrease as climate warming intensifies (Fox-Kemper, et al., 2021; Siirila-Woodburn et al., 2021). \par

No single sensor currently has the ability to directly measure SWE from space at a spatial scale suitable for water supply forecasting in mountain environments. Our ability to track other snowpack properties like snow-covered area (SCA), snow grain size, and snow albedo are mature and continually advancing with the launch of new optical sensors. However, the ability to accurately measure and monitor changes in SWE at required spatial and temporal resolution remains a challenge (A. W. Nolin, 2010; Bormann et al., 2018). While retrospective SWE reconstruction techniques have been shown to be accurate (Cline et al., 1998; Durand et al., 2008; Guan et al., 2013; Margulis et al., 2016; Molotch & Margulis, 2008; Raleigh & Lundquist, 2012; Rittger et al., 2016), they cannot help with real-time seasonal water supply forecasting. \par

This calls for the use of innovative remote sensing techniques, computational tools, and data science methods to advance our ability to measure mountain snowpacks across a range of spatial and temporal scales (Dozier, 2011). The planned NASA-ISRO SAR Mission (NISAR) (to be launched in late 2023) will provide InSAR data at both L-band (20 cm) and S-band (10 cm) wavelengths. InSAR has shown promise in tracking changes in SWE (Deeb et al., 2011; Guneriussen et al., 2000; Rott et al., 2003) but lacked the proper orbital pattern, sufficient senor wavelength, and algorithmic capacities for operational use in SWE monitoring. \par

Differential InSAR has traditionally been used in the Geology community to track movement of the Earth’s surface. The applications include tectonic events, landslides, ground subsidence, and volcanic dome expansion (Rosen et al., 2000). These processes are either event-based, whereas the event happens and the process stabilizes soon after or the surface movement is slow enough where a few measurements a year is sufficient.
Snow is the most ephemeral naturally occurring material on the Earth’s land surface. Snowpack properties such as depth, density, and SWE can significantly change on the time scale of seconds to minutes depending on atmospheric properties such as wind speed, incoming solar radiation, and precipitation intensity. In order to accurately capture the variations in SWE using InSAR, we must continue to develop the InSAR workflow to better capture the spatial and temporal variation in snowpack properties.
The progression of SWE monitoring capabilities will require data fusion-based approaches by leveraging the strengths of optical snow cover monitoring in combination with both InSAR and model-based data assimilation approaches (Durand et al., 2021). We will be testing data fusions approaches for InSAR and optical data, as well as utilizing a data assimilation based SWE reconstruction (Margulis et al., 2016) within the scope of this work.


\bibliographystyle{apalike}
\setstretch{1}
\bibliography{ch1.bib}
\setstretch{1.5}
