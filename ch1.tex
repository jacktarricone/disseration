
%==============================================================================2
%==============================================================================
%==============================================================================
%==============================================================================
\hypertarget{ch1}{%
\chapter{Introduction}\label{ch1}}


It is widely known that snow is a critical water resource for the western United States. 
*** However, our ability to accurately measure and monitor changes in snowpack, specifically its water equivalent, is challenged in mountain regions. Station-based measurements of snow water equivalent (SWE) are sparse, and spaceborne remote sensing does not yet have the capacity to directly measure SWE at the spatial resolution needed for water management applications. ***
In the western US, the majority of precipitation in mountainous regions falls in the winter as snow \citep{serrezeCharacteristicsWesternUnited1999}, where it is stored in the mountain snowpack “water towers” until it begins to melt in the spring \citep{immerzeelImportanceVulnerabilityWorld2020,viviroliIncreasingDependenceLowland2020}. Snow melt runoff provides as much as 70~\% of the total water for areas near mountains \citep{liHowMuchRunoff2017} while filling reservoirs, irrigating agricultural fields, and providing habitat for aquatic species (**change) \citep{yarnellEcologyManagementSpring2010}. 

Measuring the spatial and temporal distribution of snow water equivalent (SWE) in the world’s mountains at scales fine enough for basin-scale water resource management applications is the preeminent unsolved challenge facing snow hydrology \citep{dozierEstimatingSpatialDistribution2016}. Despite the critical importance of these measurements, the ability to operationally monitor SWE from space at basin scales remains elusive \citep{lettenmaierInroadsRemoteSensing2015}. These challenges call for the use of innovative remote sensing techniques, synergistic multisensor approaches, modeling and computational tools, and data science methods to advance our ability to measure mountain snowpacks across a range of spatial and temporal scales \citep{dozierMountainHydrologySnow2011}. 
**add refernce to SNSR here***?

To address this issue, the U.S. National Academy of Sciences 2017--2027 Decadal Survey for Earth Science and Applications from Space designated SWE and snow depth as one of the ``Most Important” observational priorities \citep{nationalacademiesofsciencesengineeringandmedicineThrivingOurChanging2019}. Earth Science Objective H-1c states the need to globally quantify rates of snow accumulation, melt, and sublimation at $\sim$100~m resolution in mountain regions. In the coming years, NASA will launch a series of Earth-observing satellites under the new Earth System Observatory (ESO) program. The first will be the joint NASA-India Space Research Organization (ISRO) synthetic aperture radar SAR (NISAR) mission in early 2024 \citep{rosenNASAISROSARNISAR2017, kelloggNASAISROSyntheticAperture2020}. NISAR will be equipped with L-band (24~cm) and S-band (9~cm) radars, global coverage, a 12~d exact repeat orbit, and interferometric capabilities. NISAR’s designated observables include glacial and sea ice monitoring, biomass estimation, low-latency natural hazards response, and measuring tectonic and geomorphic surface deformation. While seasonal snow estimation is not one of the NISAR core mission objectives, high temporal revisit (12~d), fine spatial resolution (20--100~m), and radar’s cloud penetrating ability provide a one-of-a-kind opportunity to leverage these data for SWE applications.

The overarching goal of this dissertation is to progress our capabilities for understanding snowpack variations at a range of spatial and temporal scales. We will use two research themes to accomplish these goals: quantifying the physiographic controls and sensitivities of snow metrics in the Sierra Nevada and progressing our ability to use L-band interferometric synthetic aperture radar (InSAR) to measure SWE changes. While these two methods are distinctive in both their spatial and temporal scales, they both work towards forwarding our ability to understand and monitor the dynamics of mountain snowpack. We explore these themes in three separate chapters, which are presented as independent peer-reviewed publications (or soon-to-be). 

First, in Chapter~\ref{ch2}, we quantity the physiographic controls and various snowpack metrics in the Sierra Nevada using a novel SWE reanalysis \citep{margulisLandsatEraSierraNevada2016}. These novel SWE data generation methods demonstrate the importance of spatially distributed snowpack information for conducting innovative snow hydrology research. However, the SNSR is a historical dataset and is not produced in near-real-time. The need for low-latency, high-resolution SWE information motivates our work in Chapter~\ref{ch3}, which was recently published in \emph{The Cryosphere} \citep{tarriconeEstimatingSnowAccumulation2023a}. Here, we estimate fluctuations in SWE using a novel combination of interferometric synthetic aperture radar (InSAR) data and optically-derived fractional snow covered area (fSCA) information. Lastly, in \ref{ch4}, we aim to understand optimal optical-radar multisensor approaches for future satellite-based SWE estimation. These studies represent a sequence of work where Chapter~\ref{ch1} illustrates how large-scale high-resolution SWE information is vital for understanding the complex snowpack processes, while Chapter~\ref{ch3} and Chapter~\ch{ch4} work towards the ability to produce this SWE information with low-latency from satellite remote sensing. In the following sections, we provide an overview of each chapter presented in this dissertation.

\hypertarget{ch1-intro}{\section{Physiographic controls and sensitivities of midwinter snowpack in Sierra Nevada, CA}\label{ch1-intro}}

-info

\hypertarget{ch1-intro-1}{\section{Estimating snow accumulation and ablation with L-band interferometric synthetic aperture radar (InSAR)}\label{ch1-intro-1}}

-info

\hypertarget{ch1-intro-2{\section{Towards a multisensor optical-radar approach for snow water equivalent retrievals}\label{ch1-intro-2}}


-info

Preliminary research demonstrates the ability to monitor changes in SWE using an L-band interferometric approach \citep{guneriussenInSAREstimationChanges2001,rottSnowMassRetrieval2003,deebMonitoringSnowpackEvolution2011}. However, no single radar technique works in all snow conditions. Addressing the SWE knowledge gap and reducing uncertainties will require novel and synergistic combinations of radar sensors (multi-frequency, multi-temporal, etc.), robust algorithm development, and thorough assessment using ground-based observations.
With the upcoming launch of NISAR and other planned and proposed missions (e.g., a snow mission concept proposed to the ESE call, the Terrestrial Snow Measurement Mission (TSMM) from the Canadian Space Agency (CSA), and the European Space Agencies (ESA) Radar Observation System for Europe in L-band (ROSE-L)), the time is right for continuing the development of methods for radar remote sensing of SWE. This proposed research aims to advance SWE retrieval methods by leveraging two independent radar remote sensing techniques: L-band interferometry and Ku- and X-band volume scattering. The L-band approach uses changes in time of flight and the Ku- and X-band approach uses backscatter; but both have advantages and disadvantages. Developing capability of both techniques will allow SWE estimation in a wide range of conditions. This work directly supports NASA’s snow-focused priorities as stated in the Decadal Survey (Earth Science Objective H-1c); it advances the Ku- and X-band volume scattering approach, which is the most mature technique for global SWE retrieval;






%Measuring the spatial and temporal distribution of snow water equivalent (SWE) in the world’s mountains is the preeminent unsolved challenge facing snow hydrology1. Mountain snowpack is a crucial part of the global hydrologic cycle, providing the majority of freshwater for billions of people globally2. Understanding the accumulation and ablation of SWE is vital for water resource management, global to regional land surface modeling, flood forecasting, and agricultural viability3. Despite the critical importance of these measurements, the ability to operationally monitor SWE from space at scales relevant to basin-scale water resource management remains elusive4.

% The need to measure the distribution of snow water equivalent (SWE) from space has been well-documented for decades within the hydrologic sciences community \citep{lettenmaierInroadsRemoteSensing2015, peters-lidard100YearsProgress2018}. The most recent NASA Decadal Survey listed snow depth and SWE as target observables for future NASA missions and essential for both water resource forecasting and global land surface energy balance modeling \citep{nationalacademiesofsciencesengineeringandmedicineThrivingOurChanging2019}. The need for these measurements becomes increasingly important as the areal and temporal extent of snowpacks are expected to decrease as climate warming intensifies \citep{fox-kemperOceanCryosphereSea2021,siirila-woodburnLowtonoSnowFuture2021}. \par

% No single sensor currently has the ability to directly measure SWE from space at a spatial scale suitable for water supply forecasting in mountain environments. Our ability to track other snowpack properties like snow-covered area (SCA), snow grain size, and snow albedo are mature and continually advancing with the launch of new optical sensors. However, the ability to accurately measure and monitor changes in SWE at required spatial and temporal resolution remains a challenge \citep{nolinRecentAdvancesRemote2010, bormannEstimatingSnowcoverTrends2018}. While retrospective SWE reconstruction techniques have been shown to be accurate (Cline et al., 1998; Durand et al., 2008; Guan et al., 2013; Margulis et al., 2016; Molotch \& Margulis, 2008; Raleigh \& Lundquist, 2012; Rittger et al., 2016), they cannot help with real-time seasonal water supply forecasting. \par

% This calls for the use of innovative remote sensing techniques, computational tools, and data science methods to advance our ability to measure mountain snowpacks across a range of spatial and temporal scales \citep{dozierMountainHydrologySnow2011}. The planned NASA-ISRO SAR Mission (NISAR) \citep{rosenNASAISROSARNISAR2017, kelloggNASAISROSyntheticAperture2020} (to be launched in early 2024) will provide InSAR data at both L-band (24~cm) and S-band (10~cm) wavelengths. InSAR has shown promise in tracking changes in SWE \citep{guneriussenInSAREstimationChanges2001,rottSnowMassRetrieval2003,deebMonitoringSnowpackEvolution2011} but lacked the proper orbital pattern, sufficient senor wavelength, and algorithmic capacities for operational use in SWE monitoring. \par

% Differential InSAR has traditionally been used in the Geology community to track movement of the Earth’s surface. The applications include tectonic events, landslides, ground subsidence, and volcanic dome expansion (Rosen et al., 2000). These processes are either event-based, whereas the event happens and the process stabilizes soon after or the surface movement is slow enough where a few measurements a year is sufficient.

% Snow is the most ephemeral naturally occurring material on the Earth’s land surface. Snowpack properties such as depth, density, and SWE can significantly change on the time scale of seconds to minutes depending on atmospheric properties such as wind speed, incoming solar radiation, and precipitation intensity. In order to accurately capture the variations in SWE using InSAR, we must continue to develop the InSAR workflow to better capture the spatial and temporal variation in snowpack properties.

% The progression of SWE monitoring capabilities will require data fusion-based approaches by leveraging the strengths of optical snow cover monitoring in combination with both InSAR and model-based data assimilation approaches (Durand et al., 2021). We will be testing data fusions approaches for InSAR and optical data, as well as utilizing a data assimilation based SWE reconstruction (Margulis et al., 2016) within the scope of this work.



%%%%%%%%%%%%%%%%%%%%%%%%%%%%%%%
%%%%%%%%%%%%%%%%%%%%%%%%%%%%%%%
%%% NPP

% Measuring the spatial and temporal distribution of snow water equivalent (SWE) in the world’s mountains is the preeminent unsolved challenge facing snow hydrology1. Mountain snowpack is a crucial part of the global hydrologic cycle, providing the majority of freshwater for billions of people globally2. Understanding the accumulation and ablation of SWE is vital for water resource management, global to regional land surface modeling, flood forecasting, and agricultural viability3. Despite the critical importance of these measurements, the ability to operationally monitor SWE from space at scales relevant to basin-scale water resource management remains elusive4.

% To address this issue, the U.S. National Academy of Sciences 2017-2027 Decadal Survey for Earth Science and Applications from Space designated SWE and snow depth as one of the “Most Important” observational priorities5. Earth Science Objective H-1c states the need to globally quantify rates of snow accumulation, melt, and sublimation at ~100 m resolution in mountain regions. These snow measurements were recommended as a priority in the Earth System Explorer (ESE) mission class, allowing the opportunity to launch a SWE-focused mission. One possible configuration is a Ku-band (2 cm) and X-band (5 cm) synthetic aperture radar (SAR) volume scattering approach5, which offers a feasible pathway for producing SWE data at the spatiotemporal resolution necessary for complex mountain environments. In the coming years, NASA will also launch a series of Earth-observing satellites under the new Earth System Observatory (ESO) program. The first will be the joint NASA-India Space Research Organization (ISRO) synthetic aperture radar SAR (NISAR) mission, with a planned launch in early 2024. NISAR has L-band (24 cm) and S-band (9 cm) radars, with global coverage, a 12-day exact repeat orbit, and interferometric capabilities. NISAR’s designated observables include ice movement, biomass, natural hazards, and surface deformation. While seasonal snow detection is not one of the NISAR mission objectives, the high spatial resolution (20-100 m), frequent temporal revisit, and radar’s cloud penetrating ability provide a unique opportunity to leverage the data for estimating changes in SWE. Preliminary research demonstrates the ability to monitor changes in SWE using an L-band interferometric approach6,7. However, no single radar technique works in all snow conditions. Addressing the SWE knowledge gap and reducing uncertainties will require novel and synergistic combinations of radar sensors (multi-frequency, multi-temporal, etc.), robust algorithm development, and thorough assessment using ground- based observations.

% With the upcoming launch of NISAR and other planned and proposed missions (e.g., a snow mission concept proposed to the ESE call, the Terrestrial Snow Measurement Mission (TSMM) from the Canadian Space Agency (CSA), and the European Space Agencies (ESA) Radar Observation System for Europe in L-band (ROSE-L)), the time is right for continuing the development of methods for radar remote sensing of SWE. This proposed research aims to advance SWE retrieval methods by leveraging two independent radar remote sensing techniques: L-band interferometry and Ku- and X-band volume scattering. The L-band approach uses changes in time of flight and the Ku- and X-band approach uses backscatter; but both have advantages and disadvantages. Developing capability of both techniques will allow SWE estimation in a wide range of conditions. This work directly supports NASA’s snow-focused priorities as stated in the Decadal Survey (Earth Science Objective H-1c); it advances the Ku- and X-band volume scattering approach, which is the most mature technique for global SWE retrieval; and helps demonstrate NISAR’s potential to provide critical SWE information globally. To achieve these goals, I will accomplish the following specific objectives:

%1. Characterize and quantify uncertainties and sensitivities of L-band interferometry SWE retrievals.

% 2. Implement and assess Ku- and X-band SWE retrieval algorithms using an experimental airborne radar over key NASA snow study regions.

%Science Background
%A. Snow Remote Sensing Deficiencies
%There are considerable inadequacies in current satellite-based SWE monitoring techniques, especially in complex mountain environments where uncertainties are magnified. Passive microwave (PM) radiometers use brightness temperature to estimate SWE8. These instruments produce data at multi-kilometer scale resolutions, saturate in deeper snow (>0.8 m depth), and retrievals do not work in wet snow. These retrieval limitations make PM insufficient for mapping snow in complex heterogeneous terrain. Optical remote sensing from the Moderate Resolution Imaging Spectrometer (MODIS), Landsat 8/9, and Sentinel-2 A/B obtains information on snow-covered area (SCA) and snow albedo at various spatial and temporal resolutions9,10. Unfortunately, optical sensors only provide information about snow presence, no direct information about SWE, and are routinely occluded by cloud cover.
%The Airborne Snow Observatory (ASO) uses suborbital lidar combined with a hyperspectral sensor (380–1050 nm) to measure snow depth and albedo at high resolutions (3 m) to model SWE11. The lidar portion of this technique is not transferable to a spaceborne platform because of the high-density photons per square meter needed to generate the snow depth data. To address this issue, experimental methods are being developed utilizing very high resolution satellite stereo imagery (e.g., WorldView; Pléiades) to generate snow depth maps12. Spaceborne lidars like ICESat-2 are being tested to measure snow depth, yet currently have high uncertainties and would not provide spatially distributed data13. C-band (5 cm) SAR, the wavelength employed by the Sentinel-1 A/B constellation, has been used in multiple snow monitoring capacities. Nagler & Rott14 applied a backscatter change detection algorithm to map the extent of wet snow. More recently, Lievens et al.15,16 developed an experimental Sentinel-1 snow depth retrieval algorithm using the co-polarized and cross-polarized backscatter ratio. These new methods are promising, especially in deeper snow (>1.5 m), but the underlying physics governing the retrievals are not yet well understood.

% To address knowledge gaps in snow remote sensing and prepare for a snow-focused satellite mission, the NASA Terrestrial Hydrology Program (THP) initiated the SnowEx campaign in 2016--17. SnowEx is a multiyear effort to test experimental snow remote sensing technologies in a range of globally representative snow conditions. The campaign hopes to find optimal synergistic sensor configurations and validate them through field data collection and established remote sensing techniques like airborne lidar. The two SAR-based SWE monitoring strategies tested were Ku- and X-band volume scattering and L-band interferometry, both flown on airborne platforms.


\bibliographystyle{apalike}
\setstretch{1}
\bibliography{ch1.bib}
\setstretch{1.5}
