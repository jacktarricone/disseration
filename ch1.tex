
%==============================================================================2
%==============================================================================
%==============================================================================
%==============================================================================
\hypertarget{ch1}{%
\chapter{Introduction}\label{ch1}}


It is widely known that snow is a critical water resource for the western United States (WUS). In mountain regions, our ability to accurately measure and monitor changes in snowpack, specifically its water equivalent, remains challenged. 
Station-based measurements of snow water equivalent (SWE) are sparse, and spaceborne remote sensing does not yet have the capacity to directly measure SWE at the spatial resolution needed for water management applications. However, recent advances in remote sensing \citep{lievensSnowDepthVariability2019,tarriconeEstimatingSnowAccumulation2023a, tsangReviewArticleGlobal2022} and snow data assimilation \citep{margulisLandsatEraSierraNevada2016} provide us with new tools and opportunities to measure, monitor, and map snowpack and various spatial and temporal scales.

In the western US, the majority of precipitation in mountainous regions falls in the winter as snow \citep{serrezeCharacteristicsWesternUnited1999}, where it is stored in the mountain snowpack “water towers” until it begins to melt in the spring \citep{immerzeelImportanceVulnerabilityWorld2020,viviroliIncreasingDependenceLowland2020}. Snow melt runoff provides as much as 70~\% of the total water for areas near mountains \citep{liHowMuchRunoff2017} while filling reservoirs, irrigating agricultural fields \citep{qinSnowmeltRiskTelecouplings2022a}, forest ecosystems \citep{varholaForestCanopyEffects2010}, and habitat for aquatic species \citep{yarnellEcologyManagementSpring2010}. 

Measuring the spatial and temporal distribution of SWE in the world’s mountains at spatial scales fine enough for basin-scale water resource management applications is the preeminent unsolved challenge facing snow hydrology \citep{lettenmaierInroadsRemoteSensing2015, dozierEstimatingSpatialDistribution2016}. These challenges call for the use of innovative remote sensing techniques, synergistic multisensor approaches, modeling and computational tools, and data science methods to advance our ability to measure mountain snowpacks across a range of spatial and temporal scales \citep{dozierMountainHydrologySnow2011}. 

To further address this issue, the U.S. National Academy of Sciences 2017--2027 Decadal Survey for Earth Science and Applications from Space designated SWE and snow depth as one of the ``Most Important” observational priorities \citep{nationalacademiesofsciencesengineeringandmedicineThrivingOurChanging2019}. Earth Science Objective H-1c states the need to globally quantify rates of snow accumulation, melt, and sublimation at $\sim$100~m resolution in mountain regions. In the coming years, NASA will launch a series of Earth-observing satellites under the new Earth System Observatory (ESO) program. The first will be the joint NASA-India Space Research Organization (ISRO) synthetic aperture radar SAR (NISAR) mission in early 2024 \citep{rosenNASAISROSARNISAR2017, kelloggNASAISROSyntheticAperture2020}. NISAR will be equipped with L-band (24~cm) and S-band (9~cm) radars, global coverage, a 12~d exact repeat orbit, and interferometric capabilities. NISAR’s designated observables include glacial and sea ice monitoring, biomass estimation, low-latency natural hazards response, and measuring tectonic and geomorphic surface deformation. While seasonal snow estimation is not one of the NISAR core mission objectives, high temporal revisit (12~d), fine spatial resolution (20--100~m), and radar’s cloud penetrating ability provide a one-of-a-kind opportunity to leverage these data for snow applications.

The overarching goal of this dissertation is to advance our capabilities for understanding snowpack variations across watershed-relevant spatial and temporal scales. Two research approaches were used to accomplish this goal: quantifying the physiographic controls and sensitivities of hydrologically important snow metrics and progressing our ability to use L-band interferometric synthetic aperture radar (InSAR) to measure SWE changes. While these two approaches are distinctive, they both work towards forwarding our ability to understand and monitor the dynamics of mountain snowpack. We explore these approaches in three separate chapters, which are presented as stand-alone manuscripts. 

In Chapter~\ref{ch2}, we quantify the physiographic controls and various snowpack metrics in the Sierra Nevada using a novel gridded SWE reanalysis dataset; the Sierra Nevada Snow Reanalysis (SNSR) \citep{margulisLandsatEraSierraNevada2016}. Such mountain-range-scale high-resolution gridded snowpack data allows us to explore new and spatially-based snow hydrology questions. However, the SNSR is a historical dataset and is not produced in near-real-time (1985--2016). The need for low-latency, high-resolution SWE information motivates our work in Chapter~\ref{ch3}, which was recently published in \emph{The Cryosphere} \citep{tarriconeEstimatingSnowAccumulation2023a}. Here, we estimate both accumulation and ablation of SWE using a novel combination of Interferometric Synthetic Aperture Radar (InSAR) data and optically-derived fractional snow covered area (fSCA) information. Lastly, in Chapter~\ref{ch4}, we aim to understand optimal optical-radar multisensor approaches for future satellite-based SWE estimation. These studies represent a sequence of work where Chapter~\ref{ch2} illustrates how large-scale high-resolution SWE information is vital for understanding the complex snowpack processes, and Chapter~\ref{ch3} and Chapter~\ref{ch4} work towards the ability to produce this SWE information with from satellite remote sensing. dissertation. 



\bibliographystyle{apalike}
\setstretch{1}
\bibliography{ch1.bib}
\setstretch{1.5}
