
%==============================================================================2
%==============================================================================
%==============================================================================
%==============================================================================
\hypertarget{ch5}{\chapter{Conclusions}\label{ch5}}

Mountain snowpack will continue to change, with more uncertainty expected as global warming destabilizes hydroclimatic relationships \citep{millyStationarityDeadWhither2008}. In the snowmelt-dominated WUS, the ability to accurately predict snowmelt water resources will be paramount as annual variability is expected to rise. We must continue to grow our understanding of the spatial characteristics of mountain snowpack to adapt our water management strategies in an uncertain future. The work in this dissertation helps progress our ability to monitor snowpack using two distinct approaches: mountain hydroclimatology and radar remote sensing of snow. We had three main findings in this work. First, we showed physiography (i.e., elevation, slope, and aspect) exhibits significant control on midwinter snowmelt in the Sierra Nevada. This midwinter melt propagates into Max SWE magnitude and timing, and shows different sensitivities depending on elevation and aspect. Motivated by our work showing the spatial complexities of snowpack processes, we next used L-band InSAR to effectively estimate both snow ablation and accumulation. This finding further shows the robustness of the technique and was a substantial step forward in preparation for NISAR. Lastly, we demonstrated the moderate-resolution NDSI-based optical products are sufficient for basin-scale multisensor optical-radar SWE estimation. 

We investigated the controls and sensitivities of midwinter melt metrics in only the California Sierra Nevada. Future work should focus on ablation season metrics such as snowmelt rate and melt season length. Yet, due to positive biases for melt season dynamics, the SNSR data isn't well suited for this analysis. Additionally, this type of sensitivity analysis could be scaled to the WUS using new SWE reanalysis from \cite{fangWesternUnitedStates2022}. Continue work to develop high-resolution large-scale, physically based snow models will be crucial in continuing to advance our understanding of the physiographic effects on important snowpack processes.

For continental to global scale implementation of the L-band InSAR SWE retrievals, the two main uncertainties must be addressed: atmospheric correction in complex mountain terrain and methods for confidently identifying the correct reference phase. A near-real-time atmospheric correction method that works in all of the diverse meteorologic conditions of mountain environments will be exceedingly difficult. Snowpack variations and atmospheric signals scale will elevation, entangling these two signals together. These methods should leverage past methodological advances like the persistent scatterer (PS) \citep{ferrettiPermanentScatterersSAR2001} and the small baseline subsets (SBAS) approaches \citep{berardinoNewAlgorithmSurface2002a,yunjunSmallBaselineInSAR2019a}. However, these will need to be augmented for the near-real-time snow-focused needs.

We used L-band InSAR to estimate snowpack variations in an open meadow and lightly forest moderately steep environment. However, this technique will not work in melting conditions, steep topography, and dense canopy cover. A complete understanding of global SWE distribution will require the implementation of various spaceborne SAR wavelengths and retrieval techniques and the continued use of airborne lidar surveys. Work towards a snow-focused satellite mission --- currently being proposed at either Ku-band \citep{tsangReviewArticleGlobal2022, garnaudQuantifyingSnowMass2019} or P-band \citep{shahRemoteSensingSnow2017} --- is of the utmost importance. We must also continue to minimize the uncertainties from C-band band snow depth retrievals \citep{lievensSnowDepthVariability2019,lievensSentinel1SnowDepth2022}. Understanding the performance of these experimental spatially distributed snow datasets requires spatially distributed validation data; this is best achieved through airborne lidar \citep{painterAirborneSnowObservatory2016} or terrestrial-based lidar data.

No SAR-based SWE retrieval technique will be able to provide spatially complete data. The shorter wavelengths (Ku-band and C-band) will be significantly impacted by the forest canopy \citep{rottColdRegionsHydrology2010}. While P-band can penetrate forest canopy, the satellite would provide a few (2--6) narrow swaths ($\sim$1~km) and not spatially complete data \citep{yuehSatelliteSyntheticAperture2021}. Therefore, for spatiotemporally complete data, assimilation of the snow data into land surface models will be required for all SAR-based SWE and snow depth retrievals \citep{girottoDataAssimilationImproves2020}. Assimilation methodologies have been demonstrated at Ku-band \citep{wrzesienDevelopmentNatureRun2022, choEvaluatingUtilityActive2022}, P-band \citep{maEstimatingSpatiotemporallyContinuous2023}, and C-band \citep{girottoIdentifyingSnowfallElevation2023, brangersSentinel1SnowDepth2023}, yet not for L-band. This should be the focus of future studies.

-closing paragraph here?

\bibliographystyle{apalike}
\setstretch{1}
\bibliography{ch5.bib}
\setstretch{1.5}
